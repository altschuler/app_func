%%% Local Variables:
%%% mode: latex
%%% TeX-master: t
%%% End:

\documentclass{article}

\usepackage{fullpage}
\usepackage[utf8]{inputenc}
\usepackage{listings}
\usepackage[table]{xcolor}
\usepackage{amssymb}
\usepackage{amsmath}
\usepackage{fancyhdr}
\usepackage{lastpage}
\usepackage{parskip}
\usepackage{abstract}
\usepackage{url}
\usepackage{float}
\usepackage{enumitem}
\usepackage{fancybox}
\usepackage{amsmath}
\usepackage{graphicx}
\usepackage[bottom]{footmisc}
\usepackage{hyperref}
\usepackage{makecell}


% constants
\newcommand{\COURSE}{02257 Applied Functional Programming}
\newcommand{\TITLE}{Project 1}
\newcommand{\DATE}{January 9, 2015}


%%% Local Variables:
%%% mode: latex
%%% TeX-master: "report"
%%% End:
\newcommand{\code}[1]{\texttt{#1}}
\newcommand{\fn}[2]{\lambda{} \ #1 \mapsto{} #2 }
\newcommand{\ite}[3]{\texttt{ if } #1 \texttt{ then } #2 \texttt{ else } #3 }
\newcommand{\qcol}{\quad:\quad}
\newcommand{\qvd}{\quad\vdash\quad}
%% source / reference
\newcommand{\refgeneral}[3]{#1 (#3), \emph{#2}}
% arguments:
% author
% title
% year
% publisher
\newcommand{\refbook}[4]{\refgeneral{#1}{#2}{#3}, #4}
% arguments:
% author
% title
% year
% url
\newcommand{\refonline}[4]{\refgeneral{#1}{#2}{#3} \\\url{<#4>}}
\definecolor{Brown}{cmyk}{0,0.81,1,0.60}
\definecolor{OliveGreen}{cmyk}{0.64,0,0.95,0.40}
\definecolor{CadetBlue}{cmyk}{0.62,0.57,0.23,0}
\definecolor{lightlightgray}{gray}{0.95}
\definecolor{lightgray}{gray}{0.85}
\definecolor{sh_comment}{rgb}{0.12, 0.38, 0.18}
\definecolor{sh_keyword}{rgb}{0.37, 0.38, 0.75}
\definecolor{sh_string}{rgb}{0.06, 0.10, 0.98}
\lstset{
basicstyle=\small\ttfamily,
stringstyle=\color{sh_string},
keywordstyle = \color{sh_keyword}\bfseries,
commentstyle=\color{sh_comment},
backgroundcolor=\color{lightlightgray},
frame=single,
rulecolor=\color{lightgray},
framesep=3pt,
numbersep=5pt,
xleftmargin=10pt,
xrightmargin=10pt,
showspaces=false,
showstringspaces=false,
tabsize=4,
aboveskip=5pt,
belowskip=5pt,
lineskip=2pt,
captionpos=b,
numbers=left,
numberstyle=\tiny,
stepnumber=1,
numbersep=100pt,
breaklines,
numbersep=5pt,
breakatwhitespace=false,
showspaces=false,
showtabs=false
}
\lstdefinelanguage{llvm}{
morecomment = [l]{;},
morestring=[b]",
sensitive = true,
classoffset=0,
morekeywords={
define, declare, global, constant,
internal, external, private,
linkonce, linkonce_odr, weak, weak_odr, appending,
common, extern_weak,
thread_local, dllimport, dllexport,
hidden, protected, default,
except, deplibs,
volatile, fastcc, coldcc, cc, ccc,
x86_stdcallcc, x86_fastcallcc,
ptx_kernel, ptx_device,
signext, zeroext, inreg, sret, nounwind, noreturn,
nocapture, byval, nest, readnone, readonly, noalias, uwtable,
inlinehint, noinline, alwaysinline, optsize, ssp, sspreq,
noredzone, noimplicitfloat, naked, alignstack,
module, asm, align, tail, to,
addrspace, section, alias, sideeffect, c, gc,
target, datalayout, triple,
blockaddress
},
classoffset=1, keywordstyle=\color{purple},
morekeywords={
fadd, sub, fsub, mul, fmul,
sdiv, udiv, fdiv, srem, urem, frem,
and, or, xor,
icmp, fcmp,
eq, ne, ugt, uge, ult, ule, sgt, sge, slt, sle,
oeq, ogt, oge, olt, ole, one, ord, ueq, ugt, uge,
ult, ule, une, uno,
nuw, nsw, exact, inbounds,
phi, call, select, shl, lshr, ashr, va_arg,
trunc, zext, sext,
fptrunc, fpext, fptoui, fptosi, uitofp, sitofp,
ptrtoint, inttoptr, bitcast,
ret, br, indirectbr, switch, invoke, unwind, unreachable,
malloc, alloca, free, load, store, getelementptr,
extractelement, insertelement, shufflevector,
extractvalue, insertvalue,
},
alsoletter={\%},
keywordsprefix={\%},
}
\lstdefinelanguage{JavaScript}{
keywords={typeof, new, true, false, catch, function, return, null, catch, switch, var, if, in, while, do, else, case, break},
keywordstyle=\color{blue}\bfseries,
ndkeywords={class, export, boolean, throw, implements, import, this},
ndkeywordstyle=\color{darkgray}\bfseries,
identifierstyle=\color{black},
sensitive=false,
comment=[l]{//},
morecomment=[s]{/*}{*/},
commentstyle=\color{purple}\ttfamily,
stringstyle=\color{red}\ttfamily,
morestring=[b]',
morestring=[b]"
}
\makeatother



\lstnewenvironment{promela}[0]{
\lstset{language=C++,
,morekeywords={if, then, else, return, function, var, for, foreach, do, end, od, fi}}
}
{}

\lstnewenvironment{java}[0]{
\lstset{language=C++,
}}
{}

% pseudo code environment
\lstnewenvironment{pseudo}[0]{
\lstset{
language=java
}}
{}


\pagestyle{fancy}
\fancyhf{}
\setlength{\parindent}{0pt}
\setlength{\headheight}{15pt}
\setlength{\headsep}{25pt}
\lhead{\COURSE}
\chead{\TITLE}
\rhead{\DATE}
\cfoot{Page \thepage{} of~\pageref{LastPage}}


\title{\TITLE\\ {\large \COURSE}}
\date{\DATE}
\author{
  Markus Færevaag {\tt s123692}\\
  Simon Altschuler {\tt s123563}
}


\begin{document}
\maketitle
\vspace{10cm}
\doublesignature{Markus Færevaag}{Simon Altschuler} \\
\clearpage


\section{Introduction}
In this report we will describe what language features we have
implemented in the While language, including the required and some
additional. We will end by giving some reflections on the
project.


\section{Features}
We have made all the mandatory language features, such as procedures,
recursive procedures, arrays and conditionals. This means we can run
all the given {\tt .while} programs, which can be tested by running
the {\tt Script.fsx} file, or alternatively, running {\tt make}.

The additional language features we have implemented will be described
in the following subsections.

\subsection{Comments}
To support comments with the ``{\tt // comment}'' syntax we have added a
tokenization rule in the lexer, making it ignore everything after
``{\tt //}'', on the same line.
% \begin{fs}
% rule tokenize =
%   parse
%   | ...
%   | "//" [^ '\n']* '\n' { lexbuf.EndPos <- lexbuf.EndPos.NextLine; tokenize lexbuf }
% \end{fs}

\subsection{Infix operators}
Originally the While language only supported operators with prefix
notation, as {\tt +(1, 1)}. We have added support for infix notation,
which means one can use the standard notation {\tt 1 + 1}. This works
for all common binary operators as $+$, $-$, $*$, $/$, $=$, $<$,
$>$ and $< >$.

% First we added a tokenization rule for binary operators:
% \begin{fs}
% let operator  = ('<' | '>' | '*' | '-' | '+' | '=' | '/')+

% rule tokenize =
%   parse
%   | ...
%   | operator    { BINOP(Encoding.UTF8.GetString(lexbuf.Lexeme)) }
% \end{fs}

% Where {\tt BINOP} is defined in the parser as:
% \begin{fs}
% %token <string> BINOP

% Exp:
%   | ...
%   | Exp BINOP Exp                        { Apply($2, [$1; $3]) }
% \end{fs}
Binary operators are simply to parsed ass being applied to the two
given arguments, making it equivalent to its prefix alternative.

To maintain correct precedence we had to make the dereference operator
``{\tt !}''  non-associative, and all the binary operators
left-associative. This was done by adding the following association
rules to the parser:
\begin{fs}
%left BINOP
%nonassoc CONTOF
\end{fs}

\subsection{Array literal}
Array literals are expressions of the form {\tt [exp1, exp2, ..., exp3]}, and are essentially array constants. The semantics of the array declaration are unchanged, so {\tt let int arr[3] : [1, 2, 3] in ...} would result in a two-dimensional array

A new {\tt Value} constructor has been added, namely {\tt ArrayVal List<Exp>}. Since arrays are not {\tt SimplVal}s, we must wrap them in {\tt ArrayCnt}s and add them to the store when they are being assigned to a variable.

Empty arrays are not supported because type unification is not implemented, hence we can not type check an array without elements (see \ref{sec:type} about the type system).

\subsection{Declarations-file loading}
To ease loading declarations from a separate file (specifically {\tt ArrayUtils.while}) and to avoid the need to ``hardcode'' an import of the file, we have added a {\tt decls} keyword. It requires a filename, and will load the given file, containing a list of declarations, into the environment and store.

\begin{fs}
let decls "program/ArrayUtil"; in ...
\end{fs}

The above exposes all declarations defined in the {\tt program/ArrayUtil.while} file (relative to the cwd, not the file being interpreted) in the statement block of the {\tt let ... in ...} block.

\subsection{Type system}
\label{sec:type}
A type system lets the programmer annotate the type of each variable,
enabling the intepreter to catch type errors such as assigning
something to a variable with a different type.

We have added the simple types {\tt Int}, {\tt Bool} and {\tt String}
and compound types {\tt Ref <type>} and {\tt Array <type>}. The
intepreter checks the type of each expression during assignment and
procedure application. Currently it happens during intepretation, but
could be made static by introducing a new module that traverses the
AST and builds up a type environment. It can then verify if each
variable's type aligns with the context in which it is used.

We use the function {\tt typeOf} and {\tt typeOfContent} to get the
actual type of an expression and content node and {\tt assertType} to
verify that they match.

\section{Reflections}
We find the {\tt let/in} notation somewhat strange considering that {\tt While} is an imperative language. Usually {\tt let/in} is an {\textit expression} but here it is simply used to declare variables and procedures, which could as well have been done using the assignment operator or similar in a C-like block.



\end{document}
